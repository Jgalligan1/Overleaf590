\chapter[Overleaf]{Overleaf \\
\small{\textit{-- JGa, JGr, CS}}
\index{Overleaf} 
\index{Chapter!Overleaf}
\label{Chapter::Overleaf}}

In this chapter, we discuss the details on the configuration of the Overleaf Docker container on Digital Ocean.

\subsection{Creating the Digital Ocean Droplet}

First, we created a new droplet through the Digital Ocean dashboard by selecting Ubuntu 22.04 LTS as the operating system. We chose a plan wth 4GB RAM and 2 vCPUs, selected the New York datacenter region, and configured SSH key authentication. After creating the droplet, we connected via SSH using the assigned IP address.

\subsection{Installing Docker and Docker Compose}

We first updated the system packages and installed the required dependencies:

\begin{lstlisting}[language=bash]
apt update && apt upgrade -y
apt install -y apt-transport-https ca-certificates curl \
    software-properties-common
\end{lstlisting}

Then it was necessary to add Docker's official GPG key and repository, ensuring to specify the correct architecture (amd64) for the Intel/AMD droplet:

\begin{lstlisting}[language=bash]
curl -fsSL https://download.docker.com/linux/ubuntu/gpg | \
    gpg --dearmor -o /usr/share/keyrings/docker-archive-keyring.gpg

echo "deb [arch=amd64 signed-by=/usr/share/keyrings/docker-archive-keyring.gpg] \
    https://download.docker.com/linux/ubuntu $(lsb_release -cs) stable" | \
    tee /etc/apt/sources.list.d/docker.list > /dev/null

apt update
apt install -y docker-ce docker-ce-cli containerd.io
\end{lstlisting}

Finally, we installed Docker Compose and verified both installations:

\begin{lstlisting}[language=bash]
curl -L "https://github.com/docker/compose/releases/latest/download/\
    docker-compose-$(uname -s)-$(uname -m)" \
    -o /usr/local/bin/docker-compose
chmod +x /usr/local/bin/docker-compose

docker --version
docker-compose --version
\end{lstlisting}

\subsection{Configuring the Environment}

First, we created a dedicated directory for the Overleaf configuration and navigated into it:

\begin{lstlisting}[language=bash]
mkdir ~/overleaf
cd ~/overleaf
\end{lstlisting}

We created a \texttt{.env} file to store environment variables, which proved cleaner than inline configuration:

\begin{lstlisting}
OVERLEAF_MONGO_URL=mongodb://mongo/overleaf?directConnection=true
OVERLEAF_REDIS_HOST=redis
OVERLEAF_SITE_URL=http://138.197.20.201
OVERLEAF_APP_NAME=MyOverleaf
EMAIL_CONFIRMATION_DISABLED=true
\end{lstlisting}

\subsection{Docker Compose Configuration}

We then added the \texttt{docker-compose.yml} file with several important modifications from standard configurations. The final configuration was:

\begin{minted}[options]{yaml}
version: '2.2'
services:
    sharelatex:
        restart: always
        image: sharelatex/sharelatex:latest
        container_name: sharelatex
        depends_on:
            mongo:
                condition: service_healthy
            redis:
                condition: service_started
        ports:
            - 80:80
        volumes:
            - ~/sharelatex_data:/var/lib/overleaf
        env_file:
            - .env

    mongo:
        restart: always
        image: mongo:6.0
        container_name: mongo
        command: ["--replSet", "rs0"]
        expose:
            - 27017
        volumes:
            - ~/mongo_data:/data/db
        healthcheck:
            test: echo 'db.runCommand("ping").ok' | 
                mongosh localhost:27017/test --quiet
            interval: 10s
            timeout: 10s
            retries: 5
            start_period: 40s

    redis:
        restart: always
        image: redis:6.2
        container_name: redis
        expose:
            - 6379
        volumes:
            - ~/redis_data:/data
        healthcheck:
            test: ["CMD", "redis-cli", "ping"]
            interval: 5s
            timeout: 3s
            retries: 5
\end{minted}

\subsection{Deploying and Initializing}

We composed the containers using Docker-Compose:

\begin{lstlisting}[language=bash]
docker compose up -d
\end{lstlisting}

After waiting approximately 30 seconds for MongoDB to fully start, the MongoDB replica set was initialized, which was essential for preventing transaction-related errors:

\begin{lstlisting}[language=bash]
docker exec mongo mongosh --eval \
    "rs.initiate({_id: 'rs0', members: [{_id: 0, host: 'mongo:27017'}]})"
\end{lstlisting}

We then restarted the Overleaf container to establish the connection with the initialized replica set:

\begin{lstlisting}[language=bash]
docker compose restart sharelatex
\end{lstlisting}

\subsection{Verification}

Once we saw the runit daemon start message, we verified all containers were running properly:

\begin{lstlisting}[language=bash]
docker ps
\end{lstlisting}

The Overleaf web interface is successfully hosted at \texttt{http://138.197.20.201}, confirming the deployment was complete and functional.