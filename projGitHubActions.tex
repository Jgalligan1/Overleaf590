\chapter[Github Actions]{Github Actions \\
\small{\textit{-- JGa, JGr, CS}}
\index{Github Actions} 
\index{Chapter!Github Actions}
\label{Chapter::GithubActions}}

\section{GitHub Actions Workflow Configuration}

\subsection{Overview}

We implemented a GitHub Actions workflow to automatically compile our LaTeX document whenever changes are pushed to the repository. This workflow accomplishes three key objectives: adding a version number based on the Git commit count, compiling the LaTeX document using a GitHub runner, and saving the resulting PDF back to the repository.

\subsection{Workflow File Structure}

GitHub Actions workflows are defined in YAML files stored in the \texttt{.github/workflows/} directory of the repository. We created our workflow file at \texttt{.github/workflows/latex.yml}.

\subsection{Creating the Workflow Directory}

First, we created the necessary directory structure in our repository:

\begin{minted}{bash}
cd /var/lib/overleaf/data/user_files/Overleaf590
mkdir -p .github/workflows
\end{minted}

\subsection{Workflow Configuration}

We created the workflow file \texttt{latex.yml} with the following configuration:

\begin{minted}{yaml}
name: Build Overleaf PDF
on:
  push:
    branches:
      - master
    paths-ignore:
      - '**.pdf'
permissions:
  contents: write
jobs:
  build:
    runs-on: ubuntu-latest
    steps:
      - name: Checkout repository
        uses: actions/checkout@v4
      
      - name: Set changelist number
        id: version
        run: |
          echo "CHANGE_ID=$(git rev-list --count HEAD)" >> $GITHUB_ENV
          echo "Version: $(git rev-list --count HEAD)"
      
      - name: Update version in LaTeX file
        env:
          CHANGE_ID: ${{ env.CHANGE_ID }}
        run: |
          sed -i "s/Version:.*/Version: ${CHANGE_ID}/" dsnManual.tex || true
      
      - name: Insert git version into LaTeX
        run: |
          GIT_HASH=$(git rev-parse --short HEAD)
          echo "Inserting git hash $GIT_HASH into dsnManual.tex"
          sed -i "s/PLACEHOLDER/$GIT_HASH/g" dsnManual.tex
          
      - name: Compile LaTeX document
        uses: xu-cheng/latex-action@v3
        continue-on-error: true
        with:
          root_file: dsnManual.tex
          latexmk_use_xelatex: true
          latexmk_shell_escape: true
          args: -interaction=nonstopmode -file-line-error -f
      
      - name: Upload PDF to artifacts
        uses: actions/upload-artifact@v4
        with:
          name: dsnManual-pdf
          path: dsnManual.pdf
      
      - name: Commit PDF back to repo
        run: |
          git config user.name "github-actions"
          git config user.email "actions@github.com"
          git add dsnManual.pdf
          git diff --cached --quiet || git commit -m "Add compiled PDF version $CHANGE_ID"
          git push
\end{minted}

\subsection{Workflow Components Explained}

\subsubsection{Trigger Configuration}

The workflow triggers automatically on every push to the master branch, but ignores changes to PDF files to prevent infinite loops:

\begin{minted}{yaml}
on:
  push:
    branches:
      - master
    paths-ignore:
      - '**.pdf'
\end{minted}

\subsubsection{Version Number Generation}

The changelist number is generated using Git's commit count, which provides a unique, incrementing version number:

\begin{minted}{bash}
echo "CHANGE_ID=$(git rev-list --count HEAD)" >> $GITHUB_ENV
\end{minted}

This command counts all commits in the repository history and stores the value in an environment variable accessible to subsequent steps.

\subsubsection{Version Placeholder Replacement}

We defined a placeholder in our LaTeX master file (\texttt{dsnManual.tex}):

\begin{minted}{latex}
\newcommand{\gitversion}{PLACEHOLDER}
\end{minted}

The workflow replaces this placeholder with the actual Git commit hash:

\begin{minted}{bash}
GIT_HASH=$(git rev-parse --short HEAD)
sed -i "s/PLACEHOLDER/$GIT_HASH/g" dsnManual.tex
\end{minted}

\subsubsection{LaTeX Compilation}

The workflow uses the \texttt{xu-cheng/latex-action@v3} action to compile the LaTeX document with XeLaTeX:

\begin{minted}{yaml}
- name: Compile LaTeX document
  uses: xu-cheng/latex-action@v3
  continue-on-error: true
  with:
    root_file: dsnManual.tex
    latexmk_use_xelatex: true
    latexmk_shell_escape: true
    args: -interaction=nonstopmode -file-line-error -f
\end{minted}

Key parameters:
\begin{itemize}
    \item \texttt{latexmk\_use\_xelatex: true} -- Uses XeLaTeX engine for compilation
    \item \texttt{latexmk\_shell\_escape: true} -- Allows shell commands (required for minted package)
    \item \texttt{-f} flag -- Forces compilation to continue despite warnings
    \item \texttt{continue-on-error: true} -- Prevents workflow failure on LaTeX warnings
\end{itemize}

\subsection{Committing and Deploying the Workflow}

After creating the workflow file, we committed it to the repository:

\begin{minted}{bash}
# Stage and commit the workflow
git add .github/workflows/compile-latex.yml
git commit -m "Add GitHub Actions workflow for LaTeX compilation"

# Push to GitHub (requires Personal Access Token with workflow scope)
git push origin master
\end{minted}

% \subsection{Personal Access Token Configuration}

% Since GitHub no longer accepts password authentication, we created a Personal Access Token with the following permissions:

% \begin{enumerate}
%     \item Navigate to GitHub Settings $\rightarrow$ Developer settings $\rightarrow$ Personal access tokens
%     \item Click "Generate new token (classic)"
%     \item Select scopes:
%     \begin{itemize}
%         \item \textbf{repo} (full control of private repositories)
%         \item \textbf{workflow} (update GitHub Action workflows)
%     \end{itemize}
%     \item Generate and copy the token
%     \item Use token as password when pushing via HTTPS
% \end{enumerate}

\subsection{Verification}

After pushing the workflow, we verified its execution:

\begin{enumerate}
    \item Navigate to the repository on GitHub: \url{https://github.com/Jgalligan1/Overleaf590}
    \item Click the \textbf{Actions} tab
    \item View the workflow run and check for successful completion
    \item Download the compiled PDF from the Artifacts section or view it directly in the repository
\end{enumerate}

% \subsection{Workflow Automation}
