\chapter{Jenkins Testing \\
\small{\textit{-- JG, CS}}}
\index{Jenkins} 
\index{Chapter!Jenkins Testing}
\label{Chapter::Jenkins}

A Continuous Integration (CI) pipeline was created using Jenkins to automatically build and test a Python project hosted on GitHub. The process began by setting up a local Jenkins container through Docker and connecting it to a public GitHub repository. A declarative \texttt{Jenkinsfile} was written to define the pipeline stages, including dependency installation, test execution, and report generation.

The pipeline automatically cloned the repository, installed dependencies from \texttt{requirements.txt}, and executed unit tests using \texttt{pytest}. Jenkins then displayed the results in the console output, indicating that two tests passed and one failed intentionally to confirm that the CI process detects code errors correctly.

This setup demonstrates how Jenkins integrates with GitHub to automate testing and ensure code reliability after every change.

\begin{figure}[h!]
    \centering
    \includegraphics[width=\linewidth]{png/JenkinsPart1.png}
    \caption{Jenkins environment creation}
\end{figure}

\begin{figure}[h!]
    \centering
    \includegraphics[width=\linewidth]{png/JenkinsPart2.png}
    \caption{Jenkins file creation}
\end{figure}

\begin{figure}[h!]
    \centering
    \includegraphics[width=\linewidth]{png/JenkinsBuilds.png}
    \caption{Jenkins overall testing}
\end{figure}

\begin{figure}[h!]
    \centering
    \includegraphics[width=\linewidth]{png/JenkinsPart3Failure.png}
    \caption{Jenkins success testing}
\end{figure}

\begin{figure}[h!]
    \centering
    \includegraphics[width=\linewidth]{png/JenkinsPart3Success.png}
    \caption{Jenkins failure testing}
\end{figure}

